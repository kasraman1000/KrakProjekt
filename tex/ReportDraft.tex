\documentclass[a4paper,10pt,titlepage]{article}
\usepackage{graphicx}
\begin{document}
	\tableofcontents
	\newpage
	\section{Preface}
		This project and report is made by Jakob Bang Helvind, Kasra Tahmasebi Shahrebabak, Mark Thorhauge, Filip Hjermind Jensen, Lars Yndal Sørensen og
		Anders Brorup Jørgensen. It was written for the course ’First-Year Project’ on the bachelor ”Software Development” in the Spring of 2012 at the
		IT-University of Denmark. 
		\\All work done in the project are well-documented in the following report.
	
	\section{Background and Data}
		length : 0.5 - 1 side\\
		For this project we have been given the task of designing and creating a java-written program 
		for the exam in the course 'First-Year-Project' on 2nd semester in the bachelorprogram. The basic idea of the 
		program is, that it should use some existing mapdata of Denmark to visualize a map of Denmark and let the user be
		able to search and see a given route. 
		\\This dataset is from the danish mapprovider Krak.dk and it consists of two text files. A file with all the points 
		and a file with all connections between these points.  
		\\The application should meet the following requirements:
		\\	- Draw all roads in the dataset
		\\	- The map-visualization should adjust acordingly to changes of the windows size
		\\	- The user should be able to navigate in the visualization
		\\	- The application should have a minimum of clutter and therefore be able to set a priority of the roads
		\\	- Be able to compute and show the shortest path between two points given by the user
		\\	- The application should have a logical, consistent and reasonable fast and responsive user interface\\
		
		In other words, a user should be able to see a map of Denmark with a selective amount of major roads, then explore
		 this map by zooming in and out, and otherwise navigating around. A user should also be able to select two adresses,
		 and get the shortest route between these two points calculated and shown on the map, with the map cut out to a
		 fitting view of the route.
	\section{Problemanalysis}
		\subsection{Requirements}
		length : 1-3 sider\\
		Analyse af de krav, som er beskrevet i foregående afsnit\\
		Hvad skal vi have med i vores program, for at vi kan indfri disse krav?\\
		Hvilke ting vil i have fokus paa - hastighed, hukommelse, brugervenlighed, udvidelsesmuligheder etc.?
		
		\subsection{Thoughts toward design}
			length : 0.5 side\\
			Opsummering af foregående afsnit omkring analyse af samt en præcisering af kravene. Men dette afsnit, skal bare bruges som indledning. 
			\subsubsection{Datastructure}
				Mark skriver dette for KDtræ datastruktur
				length : 1 - 2 side\\
				Hvilke generelle overvejelser er skal der tager i forbindelse med valg af datastruktur? \\
				Hvor mange har vi brug - hvilke krav er der til hver enkelt?\\
				Hvorfor har vi valgt lige netop denne datastruktur - Hvilke af vores krav indfrier de?\\
				Alternativer - Fordele / ulemper?\\

				The type of data to be stored, should be considered as much as the datastructure. Either the data could be roads with two points, a type and a priority, or it could be nodes with a point and references to road 											objects that it represents. As we want to draw roads and not nodes, it would be the most memory saving decision to store roads. However roads impose a complex problem, that is when searching for them(a range search) they should be returned when 				one or two of their coordinates are within the range. When storing roads in a grid file or a kd-tree, a road would appear twice because it has two points, making it very likely to return it twice when doing a range search. A HashSet or a similar data 						structure would solve this problem, but still slowing down the searching process. To avoid this complexity, roads could be referenced at only one of their points. This removes the possibility to return the same road twice, but imposes another problem. 					The illustration below shows roads, which are stored as points(the circles at the end of the line) and a range(the thick rectangle). When
				\includegraphics{../../../Desktop/epsRoadDS} searching in this range, all the roads should be returned. With roads referenced as a single point, R2 would not be 					returned, but R1 would. R3 would never be returned, even when referenced by two points. To solve this, an extra part of the map could be loaded, increasing the width and height of the rectangle by x, where x is the longest road in the map. Of 						course this would be inefficient if the roads’ length is not systematically broken down, and this would require the data to be restructured. To be able to guarantee that all roads within an area are shown, we will need to take this approach for that 						particular solution.
				The node has references to the roads that it is a part of. Storing nodes in the data structure has the same problem as storing roads referenced by two points. The difference is that nodes require more memory because an object is created for each 					one, but it makes the datastructure contain less elements. As there are fewer nodes than road endpoints, it makes the kd-tree(discussed later in this chapter) more shallow, and therefore faster. 
				We had several requirements for our data structure. First of all, it should be faster than linear searching through the keys of the objects and the improvement should be worth the time it takes to build the data structure when the application starts. 					When looking at smaller parts of the map the improvement should be most significant. We chose the kd-tree because its binary structure and adaptive construction makes it efficient to access data, no matter how much it clumps(how numerically close 					the keys are to each other). While grid-file is potentially faster, the kd-tree is stable and does not depend on bucket size or other factors such as cluttering. Also one of the kd-tree’s weaknesses is to return data from a rectangular query, which our 					aspect ratio does not allow, making it very suitable for our problem.


				
			\subsubsection{Algorithms}
				length : 1 - 2 side\\
				Hvilke generelle overvejelser er der ved valg af algoritmer?\
				Hvad skal vi bruge den/dem til i vores program - Hvilke problem løser de/Hvilke er vores krav indfrier de?\\
				Hvilken algoritme har vi valgt? - Hvorfor lige netop denne?\\
				Alternativer?\\
								
			\subsubsection{User Interface}
				length : 1 - 2 side\\
				Hvilke krav har vi til brugerfladen??\\
				Hvorfor bruger vi JS/Browser og ikke java Swing??\\
				
				
		\subsection{Summary}
			length : 0.5 - 1 side\\
			Opsummering\\
			Kan vi indfri vores krav/maal for programmet, ved at bruge de ting, som vi har beskrevet i det her afsnit??\\
			
	\section{Implmentation}
		length : 0.5 - 1 side\\
		Her beskrives kort hvad der kommer til at være i det kommende afsnit..
		
		\subsection{Flow description}
			length : 0.5 - 1 side\\
			Beskrivelse af flowet gennem programmet\\
			UML/Klasse diagram\\
			
		\subsection{Classes and responsibility}
			length : 0.5 - 1 side\\
			Kort beskrivelse af hver klasse plus deres ansvar\\
			
	\section{Technical Description}
		length : 0.5 - 1 side\\
		Beskrivelse af nogen af de meget tekniske og 'tunge' metoder\\
		
	\section{User Manual}
		length : 0.5 - 1\\
		Er der noget brugeren skal goere for at programmet kan koere? - koder, filer, plugin?\\
		Hvilken browser skal der bruges?\\
		Hvordan skal brugeren koere programmet?\\
		
	\section{Testing}
		length : 0.5 - 1 side\\
		Generelle overvejelser omkring test\\
		Har vi fokuseret på noget specielt i forbindelse med vores test?\\
		
		\subsection{Whitebox}
			length : 0.5 - 1 side\\
			Vi laver kun det her paa en del af programmet - hvorfor har vi valgt lige netop denne del?\\
			
		\subsection{Blackbox}
			
		\subsection{JUnit}
			
		\subsection{Errors, bugs and list of needs}
			length : 1 - 2 sider\\
			Har vi nogle fejl/bugs - Hvilke?\\
			Hvorfor er de fejl/bugs i programmet - kan vi fjerne dem?\\
			Hvad har de af konsekvenser for programmet?\\
			Hvis vi havde mere tid, hvad ville vi så have brugt den på - fejl/bugs kontra mangler??\\
			
		\subsection{Results}
			length : 0.5 side\\
			Forventningstabeller?\\
			Hvad kan vi bruge de her test til i forhold til programmet?\\
			
	\section{Product conclusion}
		length : 1 - 2 sider\\
		Har vi indfriet de krav til programmet, som vi omtalte i tidligere afsnit?\\
		Hvad ville vi goere anderleds, hvis vi havde muligheden?\\
		Mangler der noget, for at vi kan sende det her program paa gaden?\\
		
	\section{Process description}
		lenght : 1 - sider\\
		Hvordan er samarbejdet gaet i gruppen?\
		Hvad kunne vi goere for at faet et bedre samarbejde?\
		Hvad/hvilke ting var gode/daarlige i gruppen?\
		Har vi brugt nogle vaerktoejer til at styrke vores samarbejde? Kunne vi have brugt nogen?\
		
	
	\section{Final thoughts}
		length : 0.5 - 1 side\\
		Skal bare lige runde projektet af..
	
	\section{Appendix}
		
		
\end{document}